\chapter{Conclusion}

Using data from the Sloan Digital Sky Survey, we analyzed photometric observations of some $46,000$ quasars. We first preformed principle component analysis on a subset of the features present in our dataset, finding that we could reduce our data to two primary components while retaining over $90\%$ of the inherent variation in the data. While this greatly improves the visualizability of the quasar observations, we lose a nice physical interpretation for what each component means. Next, we preformed regression on the $r - i$ color versus redshift ($z$) to attempt to find a predictor for color based on the redshift value of the quasar. We used residual analysis and the Bayesian/Akaike Information Criterions to determine what degree of polynomial best fit the data, as well as subset cross-validation to find that polynomials of degree $11$ or $12$ best model our data. We compare this to the graphs of fits found by \citeauthor{wu2003} in 2003. Finally, we preform hypothesis testing on the distribution of $r$ band magnitude observations, testing whether the brightness is likely drawn from a Gaussian parent distribution. Although an initial histogram plot of the observations made the data appear normal, the Anderson-Darling and Shapiro-Wilk tests indicated that there is sufficient evidence to reject the normality of the data, which is corroborated by examining a Quantile-Quantile plot of the data. By refining the bin width of the histogram using Scott's rule, we observed the presence of a distinct subpopulation of quasars with higher brightness measurements than the rest of the population. We then plotted histograms of the brightness measurements in all other colors bands and observe the trend that the subpopulation grows more distinct as the color band gets redder.

Areas for further investigation of this dataset are plentiful. My first idea is to look more closely at the presence of the subpopulations in band brightness observations, and attempt to identify the distinct parent distributions which combine to form the bimodal sample distribution observed in the data. I would also like to understand the physical interpretation for why these measurements are bimodal, and why increasing the redness of the passband differentiates between the two populations.